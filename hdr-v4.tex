%%%%%%%%%%%%%%%%%%%%%%%%%%%%%%%%%%%%%%%%%%%%%%%%%%%
%% LaTeX book template                           %%
%% Author:  Amber Jain (http://amberj.devio.us/) %%
%% License: ISC license                          %%
%%%%%%%%%%%%%%%%%%%%%%%%%%%%%%%%%%%%%%%%%%%%%%%%%%%
%!TEX TS-program = pdfLaTeX
%!TEX encoding = utf-8
%!TEX spellcheck = fr-FR

\documentclass[12pt,twoside,openright]{book}

%\usepackage{soul}

%\documentclass[a4paper,11pt]{book}
\usepackage[T1]{fontenc}
\usepackage[utf8]{inputenc}
\usepackage[francais]{babel}

\usepackage[usenames,dvipsnames]{color}

\usepackage{lmodern}
%%%%%%%%%%%%%%%%%%%%%%%%%%%%%%%%%%%%%%%%%%%%%%%%%%%%%%%%%
% Source: http://en.wikibooks.org/wiki/LaTeX/Hyperlinks %
%%%%%%%%%%%%%%%%%%%%%%%%%%%%%%%%%%%%%%%%%%%%%%%%%%%%%%%%%
\usepackage{hyperref}
\usepackage{graphicx}
\usepackage{pdfpages}
\usepackage{amsmath}
\usepackage{amssymb}
\usepackage{a4}
\usepackage{indentfirst}
\usepackage{fancyhdr}
\usepackage{varioref}
\usepackage{makeidx}

\usepackage{mslapa}

\usepackage{ulem}

%\usepackage{apacite}
%\usepackage[longnamesfirst,nonamebreak]{natbib}




%%%%%%%%%%%%%%%%%%%%%%%%%%%%%%%%%%%%%%%%%%%%%%%%%%%%%%%%%%%%%%%%%%%%%%%%%%%%%%
%%%%%%%%%%%%%%%%%%%%%%%%%%%%%%%%%%%%%%%%%%%%%%%%%%%%%%%%%%%%%%%%%%%%%%%%%%%%%%

\pagestyle{fancy}

\addtolength{\headwidth}{\marginparsep}
\addtolength{\headwidth}{\marginparwidth}
\renewcommand{\chaptermark}[1]{\markboth{#1}{}}
\renewcommand{\sectionmark}[1]{\markright{\thesection\ #1}}
\fancyhf{}
\fancyhead[LE,RO]{\bfseries\thepage}
\fancyhead[LO]{\bfseries\rightmark}
\fancyhead[RE]{\bfseries\leftmark}
\fancypagestyle{plain}{
\fancyhead{} % get rid of headers
\renewcommand{\headrulewidth}{0pt} % and the line
}


%% Apalike hyphenation %%%
%\let\oldbibitem=\bibitem
%\renewcommand{\bibitem}[2][]{\oldbibitem[#1]{#2}\newline}

%%% Margins %%%
\voffset -1.04cm
\textheight 23cm
\hoffset -1in
\evensidemargin 2.5cm
\oddsidemargin 2.5cm
\textwidth 16cm

%%%%%%%%%%%%%%%%%%%%%%%%%%%%%%%%%%%%%%%%%%%%%%%%%%%%%%%%%%%%%%%%%%%%%%%%%%%%%%
%%%%%%%%%%%%%%%%%%%%%%%%%%%%%%%%%%%%%%%%%%%%%%%%%%%%%%%%%%%%%%%%%%%%%%%%%%%%%%

%\textwidth16cm
%\textheight23cm
%\oddsidemargin0,5cm
%\evensidemargin0,5cm
%\topmargin-1cm
%\parskip0,5cm
%\parindent0cm



%%%%%%%%%%%%%%%%%%%%%%%%%%%%%%%%%%%%%%%%%%%%%%%%%%%%%%%%%%%%%%%%%%%%%%%%%%%%%%%%
% 'dedication' environment: To add a dedication paragraph at the start of book %
% Source: http://www.tug.org/pipermail/texhax/2010-June/015184.html            %
%%%%%%%%%%%%%%%%%%%%%%%%%%%%%%%%%%%%%%%%%%%%%%%%%%%%%%%%%%%%%%%%%%%%%%%%%%%%%%%%
\newenvironment{dedication}
{
   \cleardoublepage
   \thispagestyle{empty}
   \vspace*{\stretch{1}}
   \hfill\begin{minipage}[t]{0.66\textwidth}
   \raggedright
}
{
   \end{minipage}
   \vspace*{\stretch{3}}
   \clearpage
}

%%%%%%%%%%%%%%%%%%%%%%%%%%%%%%%%%%%%%%%%%%%%%%%%
% Chapter quote at the start of chapter        %
% Source: http://tex.stackexchange.com/a/53380 %
%%%%%%%%%%%%%%%%%%%%%%%%%%%%%%%%%%%%%%%%%%%%%%%%
\makeatletter
\renewcommand{\@chapapp}{}% Not necessary...
\newenvironment{chapquote}[2][2em]
  {\setlength{\@tempdima}{#1}%
   \def\chapquote@author{#2}%
   \parshape 1 \@tempdima \dimexpr\textwidth-2\@tempdima\relax%
   \itshape}
  {\par\normalfont\hfill--\ \chapquote@author\hspace*{\@tempdima}\par\bigskip}
\makeatother


% Babel ``Sommaire'' à la place de ``table des matières''
\renewcommand{\contentsname}{Sommaire}


%%%%%%%%%%%%%%%%%%%%%%%%%%%%%%%%%%%%%%%%%%%%%%%%%%%
% First page of book which contains 'stuff' like: %
%  - Book title, subtitle                         %
%  - Book author name                             %
%%%%%%%%%%%%%%%%%%%%%%%%%%%%%%%%%%%%%%%%%%%%%%%%%%%

% Book's title and subtitle
\title{\Huge \textbf{Apprentissage dans les architectures cognitives}  \\ \vspace{1cm} \huge Contributions pour l'informatique et les neurosciences \\ \vspace{1cm}}
% Author
\author{\textsc{Emmanuel Daucé}}%\thanks{\url{www.example.com}}}


\begin{document}

\frontmatter
\maketitle

%%%%%%%%%%%%%%%%%%%%%%%%%%%%%%%%%%%%%%%%%%%%%%%%%%%%%%%%%%%%%%%
% Add a dedication paragraph to dedicate your book to someone %
%%%%%%%%%%%%%%%%%%%%%%%%%%%%%%%%%%%%%%%%%%%%%%%%%%%%%%%%%%%%%%%
%\begin{dedication}
%Dedicated to Calvin and Hobbes.
%\end{dedication}

%%%%%%%%%%%%%%%%%%%%%%%%%%%%%%%%%%%%%%%%%%%%%%%%%%%%%%%%%%%%%%%%%%%%%%%%
% Auto-generated table of contents, list of figures and list of tables %
%%%%%%%%%%%%%%%%%%%%%%%%%%%%%%%%%%%%%%%%%%%%%%%%%%%%%%%%%%%%%%%%%%%%%%%%
\tableofcontents
%\listoffigures
%\listoftables

\mainmatter

%\section*{Remerciements}
%\begin{itemize}
%\item A special word of thanks goes to Professor Don Knuth\footnote{\url{http://www-cs-faculty.stanford.edu/~uno/}} (for \TeX{}) and Leslie Lamport\footnote{\url{http://www.lamport.org/}} (for \LaTeX{}).
%\item I'll also like to thank Gummi\footnote{\url{http://gummi.midnightcoding.org/}} developers and LaTeXila\footnote{\url{http://projects.gnome.org/latexila/}} development team for their awesome \LaTeX{} editors.
%\item I'm deeply indebted my parents, colleagues and friends for their support and encouragement.
%\end{itemize}
%\mbox{}\\
%\mbox{}\\
%\noindent Amber Jain \\
%\noindent \url{http://amberj.devio.us/}

%%%%%%%%%%%%%%%%
% NEW CHAPTER! %
%%%%%%%%%%%%%%%%


%%%%%%%%%%%%%%%%%%%%%%%%%%%%%%%%%%%%%%%%%%%%%%%%%%%%%%%%%%%%%%%%%%%%%%%%%%%%%%%%%%%%%%%%%%%%%%%%%%%%%%%%%%%%%%%%%11
%%%%%%%%%%%%%%%%%%%%%%%%%%%%%%%%%%%%%%%%%%%%%%%%%%%%%%%%%%%%%%%%%%%%%%%%%%%%%%%%%%%%%%%%%%%%%%%%%%%%%%%%%%%%%%%%%%%
%%%%%%%%%%%%%%%%%%%%%%%%%%%%%%%%%%%%%%%%%%%%%%%%%%%%%%%%%%%%%%%%%%%%%%%%%%%%%%%%%%%%%%%%%%%%%%%%%%%%%%%%%%%%%%%%%%%
%%%                                            ##                                                               %%% 
%%%                                          ####                                                               %%%  
%%%                                            ##                                                               %%% 
%%%                                            ##                                                               %%% 
%%%                                            ##                                                               %%%    
%%%                                            ##                                                               %%% 
%%%                                          ######                                                             %%% 
%%%%%%%%%%%%%%%%%%%%%%%%%%%%%%%%%%%%%%%%%%%%%%%%%%%%%%%%%%%%%%%%%%%%%%%%%%%%%%%%%%%%%%%%%%%%%%%%%%%%%%%%%%%%%%%%%%%
%%%%%%%%%%%%%%%%%%%%%%%%%%%%%%%%%%%%%%%%%%%%%%%%%%%%%%%%%%%%%%%%%%%%%%%%%%%%%%%%%%%%%%%%%%%%%%%%%%%%%%%%%%%%%%%%%%%
%%%%%%%%%%%%%%%%%%%%%%%%%%%%%%%%%%%%%%%%%%%%%%%%%%%%%%%%%%%%%%%%%%%%%%%%%%%%%%%%%%%%%%%%%%%%%%%%%%%%%%%%%%%%%%%%%%%


\chapter{Introduction}


%%%%%%%%%%%%%%%%%%%%%%%%%%%%%%%%%%%%%%%%%%%%%%%%%%%%%%%%%%%%%%%%%%%%%%%%%%%%%%%%%%%%%%%%%%%%%%%%%%%%%%%%%%%%%%%%%%%
%%%%%%%%%%%%%%%%%%%%%%%%%%%%%%%%%%%%%%%%%%%%%%%%%%%%%%%%%%%%%%%%%%%%%%%%%%%%%%%%%%%%%%%%%%%%%%%%%%%%%%%%%%%%%%%%%%%
%%%%%%%%%%%%%%%%%%%%%%%%%%%%%%%%%%%%%%%%%%%%%%%%%%%%%%%%%%%%%%%%%%%%%%%%%%%%%%%%%%%%%%%%%%%%%%%%%%%%%%%%%%%%%%%%%%%
%%%%%%%%%%%%%%%%%%%%%%%%%%%%%%%%%%%%%%%%%%%%%%%%%%%%%%%%%%%%%%%%%%%%%%%%%%%%%%%%%%%%%%%%%%%%%%%%%%%%%%%%%%%%%%%%%%%
%%%%%%%%%%%%%%%%%%%%%%%%%%%%%%%%%%%%%%%%%%%%%%%%%%%%%%%%%%%%%%%%%%%%%%%%%%%%%%%%%%%%%%%%%%%%%%%%%%%%%%%%%%%%%%%%%%%
%%%%%%%%%%%%%%%%%%%%%%%%%%%%%%%%%%%%%%%%%%%%%%%%%%%%%%%%%%%%%%%%%%%%%%%%%%%%%%%%%%%%%%%%%%%%%%%%%%%%%%%%%%%%%%%%%%%



%\begin{chapquote}{Author's name, \textit{Source of this quote}}
%``This is a quote and I don't know who said this.''
%\end{chapquote}

% le modèle de Hopfield
% Qu'est-ce que le chaos?

% Qu'est-ce qu'un système apprenant

Le but de ce document est de fournir un aperçu étendu des questions et des approches que j'ai 
eu l'occasion d'aborder au cours des vingts dernières années, depuis mes premiers pas en recherche. 
{\color{Orange} Ce document est organisé autour des différentes contributions et productions scientifiques auxquelles
j'ai participé, dans un ordre non chronologique mais thématique.
A l'issue de ce parcours, il fournit également un aperçu d'un certain nombre de questions 
restées ouvertes, et pouvant faire l'objet de développements futurs.}

Ces différentes contributions appartiennent à un champ de recherche assez large à l'intersection
des neurosciences et de l'informatique, généralement appelé ``\textit{neurosciences computationnelles}''.
Le terme computationnel fait référence à la notion de calcul. On peut donc également parler de neuro-calcul,
bien que ce terme soit peu usité\footnote{  
sans doute pour ne pas le confondre avec le bio-calcul, 
qui concerne les opérations réalisées au niveau moléculaire, autrement dit
le fonctionnement de la cellule en tant que machine.}
\emph{Les neurosciences computationnelles 
visent donc à identifier ce qui, dans l'activité des neurones, s'apparente à un calcul; à identifier 
le type et la nature des opérations réalisées par le système nerveux}. 

{\color{Cyan}
Mon parcours en résumé...

L'organisation du document...
}

\section{Neurosciences computationnelles}

Si les principes de fonctionnement élémentaires du système nerveux central (neurones, synapses) 
commencent à être bien connus, 
le fonctionnement intégré du système dans son ensemble reste l'objet de nombreuses conjectures.
Les modèles mathématiques et les simulations informatiques permettent de proposer des hypothèses 
de fonctionnement et/ou 
de valider certaines hypothèses issues de l'observation. 
{\color{Orange} Les neurosciences computationnelles sont ainsi la branche de l'{\bf !!informatique!!} qui, d'une part, élabore des modèles de calcul 
	inspirés par les hypothèses proposées en neurosciences, et, d'autre part, propose des interprétations
	de données biologiques à partir de principes et modèles issus des sciences de l'information.}
Ainsi, l'étude du système nerveux, sous l'angle computationnel, a deux objectifs:
\begin{enumerate} 
\item explorer des mécanismes de calcul alternatifs 
\item comprendre le fonctionnement du cerveau
\end{enumerate}

\paragraph{Mécanismes de calcul alternatifs} Le premier objectif est
de s'inspirer du fonctionnement du système nerveux pour découvrir de nouveaux principes pour le calcul artificiel,
ou de nouvelles architectures, pour résoudre des problèmes qui résistent encore aux calculateurs actuels.
Mieux comprendre le cerveau, c'est donc potentiellement étendre le domaine de la science informatique,
étendre le type de données qu'elles sont capables de traiter (son, langage, images, ...), 
augmenter leur capacité à contrôler leur environnement, 
les rendre plus proches, 
en terme de fonctionnement, des humains avec lesquels elles interagissent. 

\paragraph{Fonctionnement du cerveau} Le deuxième objectif consiste 
à utiliser les connaissance informatiques et mathématiques actuelles pour mieux comprendre le système nerveux.
Il s'agit d'une certaine façon d'identifier ce qui dans l'activité nerveuse s'apparente aux algorithmes
utilisés dans des tâches computationnelles variées, comme le contrôle, l'apprentissage, le traitement des images et des données;
voir si les solutions mises au point dans des contextes d'ingénierie proches de celui face auquel se trouve
le système nerveux ont des équivalents au sein de l'activité nerveuse (ou si, au contraire, la nature 
utilise des principes et méthodes différents). L'objectif est, à partir des observations, de mieux 
identifier les rôles et les interactions à l'intérieur du système nerveux, et potentiellement
de mieux soigner les désordres et maladies neurologiques. 

{\color{Orange} Cette approche a environ une soixantaine d'années d'existence, {\bf !!!FLOU!!!et se confond parfois avec} 
les sciences cognitives
(étude des opérations de haut niveau réalisées par le cerveau), 
les neurosciences fonctionnelles (étude des rôles des différentes structures anatomiques), 
et l'apprentissage automatique (programmes s'adaptant aux données).}

{\color{Orange} Dans le cadre des neurosciences computationnelles, la question n'est pas nécessairement de
produire le meilleur algorithme, ou de résoudre une nouvelle tâche, mais de produire des
modèles permettant de mieux comprendre les opérations réalisées par le cerveau, ou de prendre
exemple sur des architectures ou des circuits neuronaux réels pour valider ou invalider des hypothèses
liées aux types d'opérations produites par le système nerveux.}



\subsection{Une nouvelle discipline?}



{\color{Purple} 
Les neurosciences computationnelles sont par définition pluridisciplinaires. 
\begin{itemize}
\item Elles reposent en premier lieu sur le 
champ des \textit{neurosciences}, qui ont connu un développement considérables au cours des quarante dernières années,
avec l'apparition de nouvelles techniques de mesure permettant d'observer le cerveau en fonctionnement à
différentes échelles. 
\item Elles reposent en second lieu sur la \textit{modélisation mathématique et informatique}, qui a également
connu un développement important lié à l'augmentation des capacités de calcul des ordinateurs, ainsi que de la
capacité à traiter des données massives à travers des techniques dites d'apprentisage automatique (\textit{Machine 
learning}). 
\item Un dernier champ de recherche, celui de la \textit{robotique autonome} et du contrôle moteur, est également impliqué, dans la mesure où le
développement d'interfaces bioniques nécessite une meilleure connaissance des mécanismes de 
contrôle à l'oeuvre chez l'homme et l'animal.
\end{itemize}
}

\paragraph{Trois niveaux d'analyse}
Selon la typologie classique proposée par David Marr \shortcite{Marr82}, 
un modèle peut appartenir à différents niveaux de généralité, à savoir le niveau computationnel (la 
logique), 
le niveau algorithmique (encodage et opérations réalisées), et
la réalisation matérielle (le réseau de neurones ou ``substrat'').   

Cette distinction en trois niveaux permet principalement de distinguer ce qui relève de la définition du problème~:
définition de la tâche, en fonction des contraintes physiques et de l'appareillage sensoriel et moteur, d'une part, et
d'autre part ce qui relève de la réalisation (algorithmique) de la tâche, c'est à dire principalement le circuit de 
traitement. C'est, en terminologie informatique, la distinction entre
l'équation à résoudre, la brique logicielle qui la résout et le circuit logique qui l'implémente.
Marr postule une relative indépendance entre les différents niveaux de description, en établissant une
claire séparation entre le substrat neuronal, le circuit de traitement et l'espace de la tâche. 

{\color{Orange}{\bf !!
	Cette séparation correspond assez bien au problème des disparités d'échelle 
	évoqué plus haut.!!} La contrainte physique du corps, engagé dans différentes tâches, à des échelles
spatiales macroscopiques et des échelles temporelles lentes d'une part.
La contrainte du circuit microscopique, constitué d'unité neuronales échangeant rapidement des signaux discrets
d'autre part.
Entre les deux, un niveau architectural, constitué de circuits imbriqués intégrant l'activité de
larges portion du cerveau, 
constituant un niveau de description relativement indépendant des deux autres.}

%XXXXXXXXXXXXXXXXXXXXXXXXXXXXXXXXXXXXXXXXXXXXXXXXXXXXXXXXXXXXXXXXXXXXXX
\paragraph{Approche ascendante (ou ``bottom-up'')}
La méthodologie ``bottom-up'' consiste à étudier les
\emph{conditions d'émergence} de fonctions de haut niveau à partir 
des caractéristiques d'un réseau de neurones plongé dans un
environnement contraignant.
Les réseaux de neurones sont en effet capables de manifester des formes d'organisation spontanée~: comportement critique, 
transitions entre différentes configurations d'activité, activité persistante et mémoire 
à court terme (rémanence) etc.
L'ajout de mécanismes de plasticité synaptique vise à exploiter 
cette richesse intrinsèque 
pour que le réseau puisse acquérir des connaissances et des compétences nouvelles.
{\color{Purple} 
A partir des constituants de la dynamique, de leurs propriétés (leur ``expressivité'') cette approche vise à 
établir les propriétés ou les fonctions qui peuvent être attendues de l'assemblage des constituants, autrement dit établir la 
``puissance'' d'un langage qui utiliserait 
l'activité des neurones comme syntaxe de base.
}

\paragraph{Approche descendance (ou ``top-down'')}
La méthodologie ``top-down'' consiste à étudier les \emph{conditions de réalisation}
d'une certaine tâche par étapes descendantes, de la contrainte générale du corps engagé dans la tâche
à la réalisation matérielle du circuit qui la résout.
L'approche descendante consiste souvent à identifier les aspects non-triviaux
d'une tâche. L'étude des temps de réponse, des courbes d'apprentissage et des
taux d'erreur permet d'explorer les caractéristiques limites d'un comportement et de mieux appréhender
les contraintes posées par système qui l'implémente physiquement.
Ce peuvent être des contraintes physiques (limites d'extension, de course, ...) mais
également des limites provenant des caractéristiques physiques du réseau de neurones sous-jacent.
L'élaboration d'un modèle descriptif prenant en compte plusieurs échelles permet alors d'émettre des hypothèses 
sur les sources (physiques ou logiques) des limitations comportementales observées. 

{\color{Violet}
\subsection{Développement}
Les neurosciences computationnelles se sont fortement développées, et ont commencé à
s'organiser en tant que champ de recherche, à partir de la fin des années 90, avec le soutien de plusieurs programmes 
nationaux et internationaux. On citera, parmi les plus actifs, le réseau
Bernstein pour les neurosciences computationnelles en Allemagne, développé depuis 2004, 
le collectif INSC (International Neuroinformatics Coordination Facility) qui tente de coordonner les 
nombreux projets de recherche de la discipline au niveau international (basé en Suède). Elles ont 
bénéficié de nombreux soutiens financiers européens (NEST, Brain scales, Blue Brain project, ...)
qui se sont récemment concrétisés par l'attribution d'un financement 'Flagship' pour le projet ``Human Brain
Project''  qui regroupe les principaux acteurs des projets précédents. 
Ce champ de recherche commence également à s'insérer dans les programmes de deuxième cycle, comme le montrent les
nombreux cours en ligne disponibles sur le Web
(voir par exemple ceux de Sébastien Seung au MIT, Andrew Ng à l'Université de Washington, etc.)

Au niveau des revues scientifiques, les neurosciences computationnelles ne bénéficient pas de la même
couverture académique que les disciplines plus anciennement implantées. 
Les revues qui ont historiquement participé à son développement sont les revues traitant des 
réseaux de neurones artificiels apparues à la fin des années 80 (Neural Networks, Neural 
Computation, Neural Computing Letters, Biogical Cybernetics, etc.).
Certaine revues plutôt dédiées au traitement du signal, comme Neuroimage, acceptent également des papiers du domaine.
Enfin, les revues de physique (Physica-D) hébergent également des papiers traitant des neurosciences computationnelles.  
Suite au développement du champ de recherche et de sa couverture relativement lacunaire par les revues scientifiques, 
sont apparues des revues en ``Open Acces'' spécialement dédiées au domaine. 
On notera en particulier PLOS-Computational Biology et Frontiers in Computational Neurosciences qui ont aidé à compenser
la relatif manque de visibilité du domaine.
Il existe enfin quelques conférences spécifiquement dédiées à ce champ de recherche, comme la conférence CNS (Computational
Neurosciences) et la conférence CoSyNe (Computational and Systems Neuroscience). Dans le domaine de la robotique bio-inspirée,
on peut citer la conférence SAB (Simulation of Adaptive Behavior) qui a eu un fort impact au début des années 2000 mais a 
fini par disparaître faute d'audience suffisante. 

\subsection{Les neurosciences computationnelles en France}
%Voir : https://web.archive.org/web/20131020013653/http://neurocomp.fr/

\subsection{Etat de l'art} Le champ de recherche s'organise autour de plusieurs pôles. 
(1) simulation neuro-réaliste : mieux comprendre le fonctionnement du neurone, et des interacions entre neurones. 
se cantonne parfois au simple neurone, à la stucture de la cellule. Modèles de plasticité synaptique, de la transmission
du signal nerveux, des échanges avec le milieu... (2) modèles conceptuels, principes de traitement de l'information,
codes neuronaux, mécanismes de mémoire et d'apprentissage;
(3) architectures neuronales, dynamique à large échelle, neurones à spikes, réseaux équilibrés (plutôt physique
théorique)...

Les principales avancées sont:
\begin{itemize}
 \item le système Visuel, Olshausen, les champs récepteurs, les architectures profondes, approche profonde; non-negative matrix factorization.
 \item dynamique stochastique, Fokker-Planck, etats de haute conductance, réseaux équilibrés
 \item généralisation des modèles physiques à l'analyse de l'imagerie... Modèles inverses. Friston
 \item STDP et codage par rang
 \item modèles de la décision et circuit de la récompense. Modèles du cervelet. Kalman.
 \item ce qui se fait en robotique. Franceschini. Japonais. MIT.
\end{itemize}


Problèmes rencontrés : foisonnement des modèles et des simulateurs informatiques. Difficulté à  reproduire les résultats. 
problème de la validation des modèles. Confrontation aux données empiriques. 

Il manque : des principes fondateurs, un langage commun, 

\section{Parcours personnel} 


\subsection{Les années 90}

Les voies d'entrée dans cette branche de la recherche sont multiples, puisqu'on y trouve à la fois des mathématiciens, 
des informaticiens, des physiciens, des biologistes, des médecins et des psychologues. En ce qui me concerne, 
j'ai suivi un parcours classique d'élève-ingénieur via le système des classes préparatoires et 
des concours de Grandes Ecoles. J'ai intégré l'Ecole Nationale Supérieure d'Electronique, d'Electrotechnique, 
d'Informatique, d'Hydraulique de Toulouse (ENSEEIHT) en Septembre 2002. 

Ma spécialisation vers les sciences cognitives s'est  tout d'abord faite avec la rencontre en début de troisième année avec Bernard
Doyon, médecin et chercheur à l'INSERM (Hôpital de Purpan) suite aux assises du Pôle de Recherche en Science 
Cognitives de Toulouse (PRESCOT), tenues à Labège, les 23 et 24 
septembre 1994. Bernard Doyon y présentait de façon très accessible l'étude de comportements chaotiques
obtenus par simulation de réseaux de neurones, en faisant le parallèle avec le fonctionnement du cerveau 
(en particulier avec l'étude des dimensions fractales des signaux électrophysiologiques pour différents états de veille
et de sommeil, développée à l'époque par l'équipe d'Agnès Babloyantz à Bruxelles, ainsi que les travaux
de Walter Freeman sur la réponse aus stimuli olfactifs sur des grilles d'électrodes chez le lapin).

Suite à une discussion, j'ai décidé de m'engager dans cette voie de recherche (simulation de réseaux de neurones
chaotiques) dans le cadre d'un stage du DEA de Representation de la Connaissance et Formalisation
du Raisonnement (RCFR), qui faisait partie des DEAs pouvant être suivis en parallèle de la troisième année d'études. 
Les cours étaient centrés sur la logique formelle, en particulier les grammaires génératives, 
les systèmes experts, les modèles linguistiques et la théorie de jeux. Les réseaux de neurones y étaient très peu voire 
pas abordés. 

Le stage était situé au sein du Département d'Etudes et de Recherche en Informatique (DERI), 
dans un laboratoire de l'Office Nationale de Recherche d'Etudes et de Recherche Aérospoatiales (ONERA) rattaché à 
l'Ecole Nationale Supérieure de l'Aéronautique et de l'espace (Sup'Aéro). 
J'ai intégré le petit groupe animé par Manuel Samuelidès, professeur à Supaéro, et Bernard Doyon, chercheur
à l'INSERM, autour des réseaux de neurones récurrents à connectivité aléatoire. Cette étude avait démarré 
au début des années 90.

Lors de mon arrivée, deux étudiants venaient de passer leur thèse sur ce thème.
D'un côté, Bruno Cessac, diplômé en physique théorique de l'université Paul sabatier, avait travaillé sur l'analyse en
champ moyen des réseaux de neurones récurrents aléatoires \cite{Ces95}. De l'autre, Mathias Quoy, diplômé de l'ENSEEIHT,
avait travaillé sur le comportement à taille finie (route vers le chaos) \cite{Doyon1993,CESSAC94}.
Mon rôle était alors principalement de continuer cette ligne de recherche,
et en particulier approfondir la question de la plasticité synaptique
et de l'apprentissage dans ce type de réseaux. 

Je connaissais alors peu de choses des systèmes dynamiques et du chaos, et mes connaissances en neurosciences se 
limitaient à une culture grand public, via quelques lectures. C'est donc principalement au travers 
de lectures de travaux scientifiques que je me suis formé à ces deux domaines.
L'ambiance de travail était très stimulante et nous avons rapidement pu produire 
de nouveaux résultats qui ont constitué mon rapport de fin d'études et de DEA \cite{Dau95}.
J'ai obteu mon diplôme d'ingénieur informaticien en juin 1995 et mon diplôme de DEA en septembre 1995. 
Ayant terminé à la deuxième place, j'ai pu bénéficier d'une bourse ministérielle qui m'a permis de 
poursuivre en thèse, toujours sous la direction de Bernard Doyon, au sein de l'équipe animée par Manuel 
Samuelidès au DERI.

Le sujet étudié en DEA, n'ayant pas été épuisé, est donc devenu mon sujet de thèse. 
Plus précisément, ma thèse portait sur l'étude d'un mécanisme de \textit{réduction} de la dynamique
par plasticité synaptique au sein de réseaux de neurones récurrents aléatoires.
Il s'agissait d'un sujet assez vaste, qui s'est progressivement orienté vers 
l'apprentissage et la reconnaisssance de séquences spatio-temporelles dans les réseaux de neurones récurrents
organisés en plusieurs couches de traitement. 

Cette thèse m'a également permis d'assister à l'essor du champ des neurosciences computationnelles au cours des 
années 1995-2000  (sans directement
travailler sur ces sujets), avec (1) les premiers neurones à impulsions (Thorpe-DYNN-PCNN)
(2) l'émergence du concept de connectivté fonctionnelle large échelle
(3) les modèles de champ neuronaux (Schöner), les concepts de champ récepteur

Dans le même temps, le champ des réseaux de neurones artificiels se rapproche de plus en plus de l'apprentissage 
statistique (qui deviendra le ``Machine Learning'') avec l'essor des machines à vecteurs supports (SVM) à la fin des
années 90.

Les résultats de la thèse ont donné lieu à de nombreuses publications et présentations en conférences. 
premier papier qui présente ces idées de la manière la plus complète est celui de Neural Networks, 
datant de 1998 \shortcite{Dau98A} ({\bf !! VOIR... !!}).
Il porte sur l'étude de la réaction (spontanée ou acquise) d'un réseau de neurones récurrent aléatoire
à des patrons (stimuli) également aléatoires.

% 1999 - Moynot - Pinaud - Samuelides

Une partie du travail, portant sur des architectures
multi-couches, a été réalisé en collaboration avec Olivier Moynot, étudiant diplômé de l'ENS Cachan, qui a entamé 
sa thèse sous la direction de Manuel Samuelidès un an après moi, et Olivier Pinaud, étudiant en de DEA 
de physique théorique, qui a effectué son stage à l'ONERA au printemps 1998. 

{\color{Orange} En collaboration avec Olivier Moynot, Manuel Samuelides et Olivier Pinaud \shortcite{Dau99A,dauce01a}, j'ai également 
travaillé sur la définition et 
l'étude du comportement à taille finie d'un modèle de réseau de neurones aléatoire (RNA) multi-populations (voir section \ref{sec:balanced}).{\bf!! à développer!!}}

La troisième série de résultats de la thèse et des publication associées \shortcite{Dau00,Dau02} porte sur une architecture 
neuronale  permettant d'implémenter un modèle de la perception des signaux spatio-temporels.
Ces résultats reposent sur une architecture de réseaux récurrents aléatoires à plusieurs couches (deux ou trois).
Le formalisme multi-couches est le même que celui utilisé dans les réseaux équilibrés (voir section \ref{sec:balanced}).
Ici les couches se distinguent non par la nature de leurs neurones mais par le rôle qu'elles prennent dans le processus
d'apprentissage : une (ou deux) couche(s) primaire(s) ayant un rôle passif (sans dynamique intrinsèque)
et une couche secondaire (ou associative) possédant une dynamique intrinsèque, analogue aux réseaux récurents aléatoires
à une population.

L'étude du mécanisme de reconnaissance par ``réduction'' de la dynamique constitue le coeur
de ma thèse,
}

  



Les recherches que j'ai menées balayent un grand nombre de modèles, des plus détaillés ({\bf !!REF!! neurones à 
impulsion} \cite{gerstner02}, {\bf !!REF!! neurones à conductance}) \cite{HH52} 
aux plus grossiers (activités de {\bf !!REF!! masses neuronales} à l'échelle
macroscopique \cite{Wilson1972}, modèles basés sur des mesures psychophysiques). 
Différentes architectures et méthodes d'apprentissage sont proposées, reposant sur différents 
paradigmes ({\bf !!supervisé} \cite{Rosen58}, {\bf !!REF!! non supervisé} \cite{Koh82}, 
{\bf !!REF!! par renforcement} \cite{SUTTON98}), 
sur des tâches de contrôle en boucle ouverte ou
en boucle fermée, et s'inspirant à des degrés divers de l'architecture du système nerveux.

{\color{Orange} Ce rapport {\bf!!évite les détails mathématiques!!}, et insiste sur la mise en perspectives des travaux présentés
dans le cadre des problématiques des neurosciences computationnelles. 
Pour les figures, formules et illustrations, il faudra se reporter au document annexe contenant  une sélection
de papiers et de contributions à des conférences scientifiques.}


%%% Ici on va expliquer le vocabulaire de base : les syst dynamiques, la plasticité, l'apprentissage, le contrôle
%%% en boucle fermée

Ce document est organisé en quatre grandes parties. 
\begin{itemize}
 \item Dynamique des grands réseaux de neurones aléatoires
 \item Plasticité synaptique et apprentissage
 \item Architectures de contrôle
 \item Projet scientifique
\end{itemize}

Les trois premières parties présentent les résultats  



%%%%%%%%%%%%%%%%%%%%%%%%%%%%%%%%%%%%%%%%%%%%%%%%%%%%%%%%%%%%%%%%%%%%%%%%%%%%%%%%%%%%%%%%%%%%%%%%%%%%%%%%%%%%%%%%%22
%%%%%%%%%%%%%%%%%%%%%%%%%%%%%%%%%%%%%%%%%%%%%%%%%%%%%%%%%%%%%%%%%%%%%%%%%%%%%%%%%%%%%%%%%%%%%%%%%%%%%%%%%%%%%%%%%%%
%%%%%%%%%%%%%%%%%%%%%%%%%%%%%%%%%%%%%%%%%%%%%%%%%%%%%%%%%%%%%%%%%%%%%%%%%%%%%%%%%%%%%%%%%%%%%%%%%%%%%%%%%%%%%%%%%%%
%%%                                                 #######                                                     %%%
%%%                                                ##     ##                                                    %%%    
%%%                                                       ##                                                    %%%     
%%%                                                 #######                                                     %%%   
%%%                                                ##                                                           %%%
%%%                                                ##                                                           %%% 
%%%                                                #########                                                    %%% 
%%%%%%%%%%%%%%%%%%%%%%%%%%%%%%%%%%%%%%%%%%%%%%%%%%%%%%%%%%%%%%%%%%%%%%%%%%%%%%%%%%%%%%%%%%%%%%%%%%%%%%%%%%%%%%%%%%%
%%%%%%%%%%%%%%%%%%%%%%%%%%%%%%%%%%%%%%%%%%%%%%%%%%%%%%%%%%%%%%%%%%%%%%%%%%%%%%%%%%%%%%%%%%%%%%%%%%%%%%%%%%%%%%%%%%%
%%%%%%%%%%%%%%%%%%%%%%%%%%%%%%%%%%%%%%%%%%%%%%%%%%%%%%%%%%%%%%%%%%%%%%%%%%%%%%%%%%%%%%%%%%%%%%%%%%%%%%%%%%%%%%%%%%%


\chapter{Dynamique des grands réseaux de neurones aléatoires}\label{chap:dyn}

Ce chapitre est consacré à l'analyse des capacités d'expression des grands réseaux de neurones. Je présente dans une première section quelques notions essentielles concernant la modélisation des neurones et des réseaux de neurones. La deuxième section présente quelques notions plus spécifiques à l'étude des grands réseaux de neurones ainsi que des propositions propres à ce mémoire. Enfin, la troisième section présente quelques résultats issus des travaux auxquels j'ai participé dans ce cadre.  [REPRENDRE CHAPEAU]

\section{Notions générales}

Cette section présente un formalisme et des définitions visant à établir un cadre commun aux
études portant sur les ``réseaux de neurones''. 
La plupart des définitions sont donc extrêmement générales, avec un niveau de détail faible.
Les références et exemples indiqués permettent au lecteur d'approfondir les notions
présentées s'il le souhaite.

\subsection{Neurones et réseaux}

Inspirés par le fonctionnement du cerveau, les réseaux de neurones sont des modèles de calculateurs 
 dans lesquels le ``programme'' (la fonction de réponse) est décrite par un réseau.
Ce réseau est constitué d'un ensemble de nœuds, qui sont les unités de calcul, et un ensemble d'arêtes,  
pondérées et orientées, %définissant l'intensité du signal
qui transportent le signal entre les différentes unités de calcul.

Il existe de nombreux modèles de neurones et de nombreux modèles de réseaux de neurones. La modélisation des processus neuronaux
repose donc sur un choix du modélisateur, qui est fonction du mécanisme qu'il souhaite étudier, des outils
d'analyse et/ou de la puissance de calcul dont il dispose. 

Je liste ci-dessous quelques propriétés et éléments de vocabulaire communs à la plupart des modèles de réseaux de
neurones, indépendamment de leur utilisation en modélisation ou en apprentissage automatique. 

\subsubsection{Activité et signal}

Chaque unité de calcul (ou neurone) est modélisée comme une fonction de réponse $f$ qui 
traite un jeu de \textit{données d'entrée} multimodal $s^\text{in}_1, ..., s^\text{in}_n$.
On peut supposer sans perte de généralité que les données d'entrée sont indexées sur l'axe temporel.
On parle alors de \textit{signal d'entrée} où $s^\text{in}_i = \{s^\text{in}_i(t)\}_{t \in \{t_0,... t_f\}}$ 
représente un jeu de données indexé sur une trame temporelle et 
$s^\text{in}_i(t)$ représente un point de mesure du $i^\text{ème}$ signal à l'instant $t$.

La sortie du neurone au temps $t_f$ est un scalaire: 
\begin{align}\label{eq:neurone_s_out}
s^\text{out}(t_f) = f(s^\text{in}_1, ..., s^\text{in}_n)
\end{align}
(on parle aussi de \textit{réponse} du neurone aux signaux d'entrée).
Un neurone est donc une unité élémentaire de traitement des données. %Un neurone est implicitement une unité de traitement \textit{non-linéaire}.

Le signal de sortie du neurone $s^\text{out} = \{s^\text{out}(t)\}_{t \in \{t_0,... t_f\}}$ est constituée 
d'une succession d'états ``hauts'' et d'états ``bas''.
Un neurone dont la sortie à l'instant $t$ est dans l'état haut est dit \textit{activé}. Un neurone 
dont la sortie à l'instant $t$ est dans l'état bas est dit \textit{inactif}. 

\emph{Exemples:}
\begin{itemize}
\item Dans le cas le plus simple des neurones binaires, 
le signal de sortie est une succession de 0 (inactif) ou de 1 (actif) \shortcite{MCu43}. 
\item Dans le cas plus complexe de neurones à impulsions, le signal de sortie est un train de potentiels d'action (PA), représenté par une 
somme de Diracs tels que $s(t) = \sum_{\hat{t} \in \mathcal{T}_\text{out}} \delta(t - \hat{t})$, où $\mathcal{T}_\text{out}$ est un ensemble
contenant les instants de décharge (voir par exemple \shortcite{gerstner02}).
%Dans le cas de neurones fréquentiels, le signal contient des grandeurs continues correspondant à la fréquence de décharge instantanée
%de la sortie.
\end{itemize}

Le signal de sortie (la suite d'états hauts et d'états bas) est aussi appelé l'\textit{activité} du neurone. Ce signal est transporté sur un axone, qui se
sépare à son extrémité en plusieurs branches se terminant par une (ou plusieurs) synapse(s). 

Les synapses sont les canaux d'entrée des neurones. 
Chaque synapse traite un signal $s_i^\text{in}$ émis par un autre neurone pré-synaptique $i$ du réseau et produit une \textit{entrée synaptique} $e(s_i^\text{in}$).
L'état \textit{interne} du neurone post-synaptique est donné par son \textit{potentiel de membrane} $V$.
La somme des entrées synaptiques agit sur la valeur de ce potentiel de membrane. 
L'intégration de la totalité des entrées peut être exprimée par une fonction de mise à jour du potentiel de membrane de la forme:
\begin{align}\label{eq:neurone_V}
V = g(s_1^\text{in}, ..., s_n^\text{in})
\end{align}
où $s_1^\text{in}, ..., s_n^\text{in}$ sont les $n$ signaux entrants.

\emph{Exemples:}
\begin{itemize}
\item Dans le cas le plus simple, $g$ est une combinaison linéaire des entrées synaptiques \shortcite{MCu43,Rosen58}.
\item Les modèles plus détaillés prennent en compte 
la fonction de transfert des synapses ainsi que les interactions non-linéaires entre les influences excitatrices et les influences inhibitrices \shortcite{DESTEXHE97}.
\end{itemize}

L'activation d'un neurone repose ensuite sur un mécanisme non linéaire de \textit{passage de seuil}.

\emph{Exemples:}
\begin{itemize}
\item Dans le modèle le plus simple \shortcite{MCu43}, l'état de sortie (0 ou 1) dépend d'un
seuil d'activation $\theta$ tel que $s=1$ (état haut) si $V>\theta$ et $s=0$ (état bas) sinon.
\item
Dans les modèles plus détaillés \shortcite{Lapicque1907,HH52}, le passage à l'état haut déclenche un mécanisme actif de réinitialisation qui ramène
le potentiel de membrane vers sa valeur de repos $V_0 < \theta$.  Ce mécanisme de réinitialisation, qui interdit
le maintien de la sortie dans l'état haut, rend l'état haut  
plus rare (et donc plus porteur d'information) que l'état bas. Il a également pour effet d'effacer 
la mémoire des données d'entrée antérieures au dernier potentiel d'action.
\end{itemize}


%% A mettre dans la partie sur la plasticité
%Un des modèles les plus étudiés en modélisation est le modèle de l'\textit{intégrateur à fuite}. 
%La valeur du potentiel dépend de l'historique des signaux d'entrée depuis le dernier PA jusqu'à l'instant $t$. 
%{\bf !! TODO  mécanisme de décision basé sur l'accumulation + figure!!}.
% La partie suivante doit commencer avec les modèles de la décision : décision bayésienne, intégrateur à seuil, décision de Hopfield,
% décision du champ neuronal, avec pour finir le modèle de Pouget (compromis dynamique)
% Préciser ensuite le mécanisme de sélection/intégration

\subsubsection{Réseau de neurones}

Plusieurs neurones connectés entre eux via des axones constituent 
un \textit{réseau de neurones}. Le schéma de connexion, sous la forme d'un graphe orienté, 
caractérise l'\textit{organisation structurelle} du réseau. 
%Selon que le graphe est acyclique, ou au contraire récurrent, .

La description du réseau est donné par une \textit{fonction de couplage},
définie le plus souvent par un tableau à double entrée $T$. Chaque neurone
se voit attribuer un indice $i \in 1,...,n$ où $n$ est le nombre de neurones dans le réseau. Le tableau
$T[i,j]$ contient alors les caractéristiques du lien du neurone $i$ vers le neurone $j$.
Un exemple de paramétrisation est le temps de propagation $\tau_{ij}$ et poids synaptique $J_{ij}$. 
Lorsque le poids synaptique est non-nul, on dit que les neurones $i$ et $j$ sont couplés. 
Si $n$ est le nombre de neurones, le nombre de paramètres du réseau est en $O(n^2)$, correspondant à un principe de transmission du signal \textit{de plusieurs à plusieurs} (un neurone envoie son signal de sortie vers plusieurs destinataires, un neurone reçoit des signaux de plusieurs émetteurs).

\emph{Notations:}
\begin{itemize} 
\item Le \textit{vecteur d'activité} $\boldsymbol{s}(t)$ (ou activité de population) est un vecteur de taille $n$ contenant l'ensemble des 
sorties à l'instant $t$~: 
$$\boldsymbol{s}(t) = (s_1^\text{out}(t), ..., s_n^\text{out}(t))$$
\item Le \textit{patron d'activité} est constituée de l'ensemble des signaux émis entre l'instant initial $t_0$  et l'instant d'observation $t$, soit: $$\boldsymbol{S}(t)=\{\boldsymbol{s}(t')\}_{t' \in [t_0, t[}$$
\end{itemize}

Les activités des différents neurones sont donc \textit{inter}dépendantes du fait des couplages. 
La \textit{fonction de réponse} du réseau $f_T$ est une fonction paramétrique 
décrite par le tableau de paramètres $T$, telle que l'activité $\boldsymbol{s}(t)$ est solution de~:
\begin{align}\label{eq:optim}
\boldsymbol{s}(t) = f(\boldsymbol{S}(t);T)
\end{align}
Il faut ici préciser, dans la mesure où les temps de transports sont supposés strictement positifs, que l'activité du réseau au temps $t$ est
dépendante de l'historique des activités précédent strictement 
l'instant $t$. L'activité d'un neurone post-synaptique $i$ au temps $t$ est dépendante des signaux pré-synaptiques
produits aux instants $t' < t$, en tenant compte du \textit{temps de transport} de ces signaux sur les axones. Plus précisément, si $j$ est le 
neurone pré-synaptique et $i$ le neurone post-synaptique, on a:
\begin{align}\label{eq:transport}
s^\text{in}_{ij}(t) = s^\text{out}_{j}(t - \tau_{ij})
\end{align}

\subsubsection{Architecture de réseau}

On parle d'\textit{architecture de réseau} pour décrire les propriétés générales du graphe.
L'architecture définit généralement le nombre de couches, le type de connexions de couche à couche, 
et éventuellement le nombre de neurones par couche.
Un exemple typique d'architecture est l'organisation séquentielle des réseaux de neurones dits
``à couche cachée'',
qu'on trouve dans les perceptrons multi-couches, les réseaux profonds, etc. [CITATIONS] 
Un autre exemple d'architecture est celle des mémoires associatives, constituées d'une ou plusieurs couches de 
neurones connectés sous la forme d'un circuit récurrent (ou réentrant) [CITATION].

{\color{Cyan} Notion de faisceau d'axones. Construction de réseaux.}

\subsubsection{Calcul distribué}

Les signaux transitant de neurone à neurone via les axones du réseau contiennent une succession d'états hauts et
d'états bas qui présentent une ressemblance  avec les signaux digitaux produits par les calculateurs 
numériques. Cette analogie de forme n'implique cependant pas que les principes de traitement et
de transformation de ces signaux soient les mêmes.

Du point de vue des calculateurs numériques, un calcul est décrit comme une séquence d'opérations élémentaires réalisées dans un certain ordre
à partir de données d'entrée. Ces opérations élémentaires peuvent être réalisées par les portes logiques
d'un circuit électronique, ou par tout mécanisme automatique capable de ségréger ses entrées sous la forme d'au moins 
deux états de sortie distincts. 
%La généralisation 
%aux calculateurs universels se fait à travers la présence de mémoires, c'est à dire d'unités capables de
%conserver un état (haut ou bas) pour une durée indéterminée. 

Dans la cas des réseaux de neurones, la mise en oeuvre d'un calcul repose sur la présence
de neurones d'\textit{entrée} soumis à des données extérieures, qui sont donc les opérandes du calcul.
C'est le cas chez l'animal des cellules \textit{sensorielles}. 
On note $\boldsymbol{I}(t) = \{\boldsymbol{I}(t')\}_{t' \in [t_0, t[}$ ce signal extérieur. 
Si le neurone $i$ est un neurone d'entrée, on a $I_i \neq 0$. Si le neurone $i$ n'est pas un neurone d'entrée,  
on a $I_i = 0$.
Le signal extérieur est généralement considéré comme indépendant de l'activité du réseau de neurones.

La \textit{réponse} du réseau de neurones au signal d'entrée (la solution du calcul) est le patron d'activité induit
 par ce signal d'entrée.
En étendant la fonction de réponse aux entrées extérieures, l'activité $\boldsymbol{s}(t)$ est la solution de~:
\begin{align}\label{eq:optim_input}
\boldsymbol{s}(t) = f(\boldsymbol{S}(t), \boldsymbol{I}(t);T)
\end{align}
On dit également 
que le réseau de neurones \textit{transforme} le signal d'entrée.

Il est possible de définir une \textit{fonction de sortie} qui \textit{décode} ce patron d'activité. 
La sortie du réseau est alors 
$$\boldsymbol{u} = h(\boldsymbol{s})$$
avec $h$ fonction de décodage (ou de ``\textit{read-out}'').

Plus globalement, la fonction d'entrée/sortie (fonction de transfert) du réseau de neurones est~:
\begin{align}\label{eq:E/S}
 \boldsymbol{u} = h(f(\boldsymbol{S},\boldsymbol{I};T))
\end{align}

La sortie $\boldsymbol{u}$ peut ainsi être interprétée comme le résultat du calcul réalisé à partir des données d'entrée $\boldsymbol{I}$. 
Dans la mesure où ce résultat est le produit de 
l'activité conjointe (et parallèle) des neurones du réseau, on se situe dans un contexte de \textit{calcul distribué}, par opposition au calcul
séquentiel centralisé réalisé par les ordinateurs traditionnels.



\subsection{Réseaux de neurones et modélisation biologique}

L'inspiration biologique conduit à étudier des modèles alternatifs au traitement
séquentiel standard. 
Dans le cadre de ce mémoire, nous regarderons en particulier 
l'idée de \textit{circuit reconfigurable}, basée sur l'existence d'une \textit{activité endogène} produite par un réseau de 
neurones \textit{récurrent}.  
Le but est de caractériser ces mécanismes, et 
d'identifier les points communs avec  
des mécanismes similaires à l'{\oe}uvre dans le cerveau. 

Les réseaux de neurones récurrents sont définis par le fait que leur graphe contient des cycles.
Ces cycles conduisent à des dépendances réciproques entre les unités du réseau. 
Le formalisme des systèmes dynamiques permet de décrire ces interactions croisées et leur effet sur les patrons
d'activité produits par le réseau.

\subsubsection{Quelques notions de la théorie des systèmes dynamiques}
L'activité du réseau au temps $t$ est le résultat de l'intégration des
potentiels d'action émis aux instants précédant $t$, en tenant compte des délais de transmission, et de la valeur courante du signal externe
(voir eq. (\ref{eq:optim_input})).

La théorie des \textit{systèmes dynamiques} repose sur un espace d'état $\mathcal{X}$, une trame temporelle 
$\mathcal{T}$ et un
flot $\phi$ qui est une application de $\mathcal{X} \times \mathcal{T}$ dans $\mathcal{X}$
définissant pour tout couple $(x,t)$ la valeur de l'état suivant:
\begin{align}\label{eq:SD}
x' = \phi(x,t)
\end{align}
La plupart des réseaux de neurones peuvent se modéliser sous cette forme, à partir du moment où 
l'état du réseau à l'instant $t$ est entièrement spécifié, en tenant compte en particulier
du potentiel de membrane et 
des différents temps de transport sur les axones. 

La \textit{trajectoire} du système sur la plage temporelle $[t_0,t_f]$ est alors définie 
par l'intégration sur le flot de la condition initiale $\boldsymbol{x}_0$.

\begin{itemize}
\item Dans le cas d'un système dit ``autonome'', on a $\phi(x,t) = \phi(x)$ et
la trajectoire du système $\{x(t)\}_{t \in [t_0,t_f], \boldsymbol{x}(t_0)= \boldsymbol{x}_0}$ 
est entièrement définie par les conditions initiales. 
\item Dans le cas d'un système dit ``non-autonome'', la dépendance temporelle
est souvent modélisée sous la forme d'un signal externe $I(t)$, soit $\phi(x,t) = \phi(x,I(t))$. 
Ce signal peut être:
  \begin{itemize}
  \item une donnée d'entrée dans le cas de modèles de traitement des données
  \item un bruit externe dans le cas de modèles stochastiques [REF?]
  \end{itemize}
\end{itemize}

{\color{Cyan}
Notions :
\begin{itemize}
 	\item notion de variable d'état définie pour tout $t$.
\end{itemize}
}

\paragraph{Systèmes dynamiques autonomes}
Dans le cas autonome...
{\color{Cyan} 

Notions :
\begin{itemize}
	\item système conservatif / dissipatif
	\item Notion d'attracteur =sous-ensemble connexe $C C \mathcal{X}$ invariant par le flot : $C = \phi(C)$.  Condition de stabilité différente si discret ou continu.
	\item relaxation du système (dynamique de relaxation). Différentes CI conduisent à différents attracteurs.
	\item paramètre de contrôle et bifurcations
	\item bassins d'attraction et séparatrices
	\item diagramme de phase
	\item paramètre d'ordre
\end{itemize}
}
\paragraph{Systèmes dynamiques non-autonomes}

Dans le cas non autonome, la trajectoire dépend à la fois des conditions initiales et du signal extérieur. 
\begin{itemize}
\item On parle dans ce cas d'une activité \textit{induite} par le signal d'entrée. 
\item Du point de vue 
computationnel, celà signifie que le même signal peut recevoir un traitement différent si les conditions initiales sont différentes. 
Sans précision des conditions initiales, de multiples opérations sont donc possibles pour un même signal d'entrée. 
{\color{Orange} On parlera de différents \textit{modes d'opération}}.
\end{itemize}


En pratique, l'équation (\ref{eq:SD}) prend souvent la forme d'une équation différentielle [REFS]
ou intégro-différentielle [REFS] qui se résout
par des méthodes numériques. 


%Dans un cadre continu, l'équation d'évolution est la suivante~:
%$$ \frac{d\boldsymbol{x}}{dt} = \Phi(\boldsymbol{x},t)$$
%Dans un cadre discret, l'équation d'évolution est~:
%$$ \boldsymbol{x}(t+1) = \Phi(\boldsymbol{x}(t),t)$$

%où $\Phi(\boldsymbol{x},t)$, où $t \in \mathcal{T}$ décrit un instant
%situé sur la trame temporelle et la variable d'état $\boldsymbol{x} \in \mathcal{X}$ décrit
%l'état du système.

\emph{Exemples:}
\begin{itemize}
\item Un exemple minimaliste est un réseau de neurones récurrent à deux neurones mutuellement connectés. 
Un tel réseau pourrait présenter deux modes : un mode ``inactif'' dans lequel les deux neurones
sont dans l'état ``bas'', et un mode ``actif'' dans lequel une activité ``haute'' locale est transmise
indéfiniment d'un neurone à l'autre. Chacun de ces deux patrons d'activité est une solution 
du graphe de contraintes. 
%\item 

\item {\color{Orange} L'exemple des réseaux de Hopfield \shortcite{Hop82} illustre bien la notion de modes multiples. 
%offre  un éclairage différent. 
%sur l'expressivité d'un substrat en mettant en 
%avant le caractère distribué (non-local) de l'activité. 
%Contrairement à l'idée classique d'un noeud de sortie comme unité d'expression, 
%(par exemple unité de séparation dans le cas des classifieurs, 
%unité de description dans le cas des cartes auto-organisées etc.), 
L'expressivité du réseau est vue sous l'angle du nombre
de patrons distribués (patterns) pouvant être obtenus en tant qu'état final (attracteur) atteint par le système
dynamique décrit par les poids (arêtes) du réseau.
La capacité s'exprime ici en nombre de \emph{modes} (configurations atteignables) 
et non en nombre de noeuds/vecteurs supports etc.
Si la capacité se trouve en principe augmentée
d'un facteur exponentiel ($2^N$ configurations distribuées possibles), 
la capacité effective est en pratique beaucoup plus réduite, 
se limitant à un nombre de modes (attracteurs) distincts en $O(N)$, 
c'est à dire de l'ordre du nombre de noeuds \shortcite{Amit1987,Tso88}.
La nature des opérations réalisables par le substrat est également différent puisqu'il s'agit principalement ici d'une
dynamique de relaxation adaptée à des opération de complétion et d'interpolation de données manquantes, implémentant un
mécanisme de mémoire dite ``auto-associative''. }
\item 

\end{itemize}


Lorsque le système est dissipatif, le flot définit un ensemble ... 

notion d'état du réseau, et de patron d'activité

axe temporel, système dynamique et conditions initiales.

équation d'évolution de la dynamique


Notions :
\begin{itemize}
 \item attracteur
 \item relaxation
 \item séparatrice
 \item paramètre de contrôle
 \item paramètre d'ordre
\end{itemize}


Dans le cadre des études inspirées par la biologie, il est fréquent d'utiliser des modèles plus simples
appelés modèles de \textit{champ moyen}.

Nous regardons en particulier le cas des réseaux de neurones 
récurrents aléatoires [A CONNECTER AU RESTE].

\subsubsection{Grands réseaux et champ moyen}


%Les modèle qui servent de fondement aux neurosciences computationnelles 
%sont les modèles de neurones à conductance [REF DESTEXHE] dont le plus connu est le modèle de Hudgkin-Huxley. 


\paragraph{Taux de décharge instantané} L'activité des neurones telle que mesurée par électrophysiologie présente
un caractère irrégulier où les temps d'émission des PA semblent obéir à un processus de Poisson [CITATION].
Un tel processus est caractérisé par sa fréquence instantanée $\nu$, qui donne le nombre de PA émis par seconde.
La grandeur $\nu$ peut être mesurée 
à travers l'observation d'un train de PA produit par un neurone particulier (nombre de PA émis divisé par
l'intervalle d'observation).

De nombreux modèles stochastiques de neurones ont été proposés qui prennent en compte le caractère aléatoire de la 
réponse des neurones biologiques. 


\paragraph{Activité de population}
Le modèle de champ moyen considère l'activité instantanée de populations de neurones $\nu(t)$, qui exprime, à un instant
donné, le nombre de neurones émettant un PA divisé par la taille de la population.
Un interprétation commune de l'activité des

\section{Notions spécifiques et propositions}
Cette section 

\subsection{Grands réseaux de neurones aléatoires}

\subsection{Modes et expressivité}

{\color{Orange}
	\paragraph{!!Propositions!!}
	
	Il est souvent possible de regrouper différents patrons d'activité selon des caractéristiques communes
	se traduisant par une valeur unique au niveau de la fonction de sortie $h$. 
	\begin{itemize}
		\item On parlera des différents \textit{modes de sortie} du réseau de neurones.
		\item L'\textit{encodage} est le processus par lequel un jeu a priori non borné de signaux 
		d'entrée se traduira sous la forme d'un nombre fini de modes de sortie.
		Chaque mode est alors le reflet, au sein du réseau, d'une \textit{classe} de signaux. 
		\item On parlera d'\textit{expressivité} pour désigner le nombre et la nature des modes de sortie produits par un réseau de neurones,
		en présence ou en l'absence de signal extérieur. 
		Le nombre et la nature de ces modes
		nous permettra de décrire son ``vocabulaire'', ainsi que le type d'opérations dont il peut être le support. 
	\end{itemize}
Notion de réponse d'un mode?  
}

\section{Contributions personnelles}

\chapter{Plasticité synaptique et apprentissage}
Ce chapitre est consacré à l'étude des mécanismes d'apprentissage dans les réseaux de neurones récurrents.  

\section{Notions générales}

\subsection{Plasticité synaptique}

La synapse biologique est l'interface permettant la communication entre les neurones.
Il s'agit de petites surfaces d'échange chimique (``boutons'' synaptiques) situés à l'extrémité de l'arborisation terminale des axones. A l'arrivée d'un potentiel d'action, les synapses libèrent des neurotransmetteurs qui agissent sur les canaux ioniques des dendrites de la cellule post-synaptique (``libèrent'' des ions), ce qui a pour effet de modifier le potentiel de membrane de la cellule post-synaptique.

L'efficacité d'une synapse dépend de plusieurs facteurs, comme la taille du bouton synaptique, la quantité de neurotransmetteurs disponibles,  ainsi que la sensibilité de la cellule post-synaptique. Dans le cadre de ce mémoire,  cet ensemble de facteurs est résumé  sous la forme d'une valeur unique $J_{ij}$ : le ``poids'' de la synapse, où  $j$ est l'index du neurone pré-synaptique et $i$ l'index du neurone post-synaptique.

La plasticité synaptique est un mécanisme biologique qui modifie l'efficacité de la synapse au cours du temps. En reprenant 
les notations de l'équation (\ref{eq:SD}), 
\begin{align}
&x' = \phi(x,J,t) \label{eq:SD-plast-x}\\
&J' = \psi(x,J,t) \label{eq:SD-plast-J}
\end{align}
où $x$ est un vecteur représentant l'état du système dynamique et $J$ une matrice représentant le graphe de connexions.
La plasticité (eq. \ref{eq:SD-plast-J})
modifie donc les caractéristiques de la fonction de réponse des neurones (eq. \ref{eq:SD-plast-x}) et vice-versa. 
Le mécanisme de plasticité introduit une interdépendance complexe entre le graphe, l'activité et le signal d'entrée s'il existe.

\subsubsection{Plasticité de Hebb}
Dans le cadre 
proposé par Donald Hebb en 1949 \cite{Heb49}, et confirmé depuis par des observations \cite{Bli73,BIPOO98}, la plasticité est essentiellement un mécanisme local dépendant des échanges entre les cellules pré et post-synaptiques. On parle de potentiation à long terme (Long Term Potentiation - LTP). Le poids $J_{ij}$ est alors une quantité qui évolue au cours du temps sous la forme :
\begin{align}
J_{ij}(t) = F(S_i(t), S_j(t), J_{ij}(t_0))
\end{align}
avec $t_0$ l'instant initial d'observation,
$S_j(t) = \{s_j(t)\}_{t \in [t_0,..,t[}$  l'activité pré-synaptique, $S_i(t) = \{s_i(t)\}_{t \in [t_0,..,t[}$ l'activité post-synaptique,
et $F$ la fonction de mise à jour des synapses.
%La plasticité est donc essentiellement un mécanisme \textit{local} qui modifie le couplage des cellules $i$ et $j$ en fonction 
%de leurs activités respectives.  

%En reprenant les notations proposées au début du chapitre \ref{chap:dyn}, la dynamique d'ensemble devient~:
%\begin{align}\label{eq:plasti-input}
%&J(t) = F(\boldsymbol{S}(t), J(t_0))\nonumber\\
%&T(t) = \{\tau_{ij},J_{ij}(t)\}_{i,j\in[1,...,n]}\\
%&\boldsymbol{s}(t) = f(\boldsymbol{S}(t), \boldsymbol{I}(t);T(t)) \nonumber
%\end{align}

%Ce jeu d'équations illustre l'interdépendance entre le mécanisme d'activation $f$ (qui ``produit'' le patron d'activité $\boldsymbol{S}$ agissant sur les synapses) et le mécanisme de plasticité $F$ (qui ``produit'' le patron d'interaction $J$ qui agit sur la fonction d'activation). 


{\color{Orange} Il existe, du point de vue biologique, des mécanismes de plasticité synaptique à court terme [REF STD] et à long terme [REF LTP]. On parle de potentiation à court terme vs. potentiation à long terme.}
Dans le cadre de ce mémoire, on considérera l'évolution des poids synaptiques comme ``lente'' par rapport à la dynamique d'activation (autrement dit que les poids synaptiques sont quasi-stationnaires sur de petits intervalles de temps).

Cette dynamique lente a un impact sur le comportement du réseau de neurones sur le long terme. Il s'agit essentiellement, selon l'idée initiale de Hebb, d'un mécanisme 
de sélection dans lequel des activités corrélées tendent à se connecter, et des activité décorrélées à se déconnecter. 
La règle de Hebb s'interprète donc généralement comme une règle qui inscrit dans le graphe la structure de covariance présente dans l'activité des neurones. Lorsque l'activité est elle-même induite par le signal d'entrée, le graphe reflète en partie les covariances présentes dans le signal.

{\color{Cyan}
Exemples (avec maths):
\begin{itemize}
	\item Co-activité
	\item Covariance
	\item STDP
\end{itemize}
}

\subsubsection{Plasticité induite}

La plasticité induite est un mécanisme  de plasticité déclenché par un signal extérieur. 
Il peut s'agir, dans un cadre biologique, d'un neurotransmetteur ou d'une entrée synaptique spécifique, i.e.~:
\begin{align}
J_{ij}(t) = F( S_i(t), S_j(t), y(t), J_{ij}(t_0))
\end{align}

Ce signal extérieur $y(t)$ est~: 
\begin{itemize}
	\item dans le cas le plus simple, un  déclencheur binaire qui ``ouvre'' ou ``ferme'' le mécanisme de plasticité;
	\item un facteur réel qui, selon son signe, conduit à une plasticité Hebbiennes ou anti-Hebbienne;
	\item un signal d'erreur transmis par des synapses spécifiques. Dans ce cas, $y$ a le rôle d'un ``méta-signal'' transmettant une consigne sur la manière dont il faut modifier le graphe de connexions;
	\item...
\end{itemize}

\paragraph{Remarque} On notera que l'application alternée d'une plasticité Hebbienne et anti-Hebbienne ouvre la voie
à l'implémentation de mécanismes d'apprentissage par renforcement, 
ainsi qu'à la modélisation des mécanismes de plasticité liés à la récompense et 
à la punition dans le cerveau \shortcite{Barto1995,Schultz1997,Gurney2001}. Ces aspects sont analysés 
plus en détail dans le chapitre suivant.

\subsection{Apprentissage}

\subsubsection{Perspective biologique}
Selon une perspective biologique et développementale, l'apprentissage
est le processus de changement comportemental, 
en relation avec ce mécanisme de plasticité. 
L'apprentissage est au sens large
l'ensemble des processus épigénétiques (historiques) inscrivant dans l'animal ou l'individu 
les éléments d'expérience qui contribuent à accroître l'adaptation de ses réponses à son
milieu.  

Cette capacité à inscrire des événements ou des faits particuliers en vue de les exploiter dans le futur est une des
propriétés essentielle du système nerveux des être vivants. 
Mieux comprendre les mécanismes biologiques qui sous-tendent cet apprentissage est un des défis majeurs pour les neurosciences computationnelles. Si la plasticité synaptique semble être le principe explicatif majeur de l'apprentissage, il reste encore de nombreuses zones d'ombre 
\begin{itemize}
	\item sur les caractéristiques précises de cette plasticité;
	\item sur les mécanismes de choix qui vont sélectionner certains signaux et certains événements plus significatifs;
	\item sur les déterminants macroscopiques des changements microscopiques et vice-versa.
\end{itemize}

\subsubsection{Perspective computationnelle}

Du point de vue computationnel, l'apprentissage automatique
(le ``Machine learning'')  consiste à confier une tâche 
de programmation à un algorithme, dit algorithme d'apprentissage. 
En d'autres termes, il faut écrire un programme \textit{capable de programmer}. 
%le réseau de neurones. 
Ce \textit{méta-programme} s'appuie sur des méta-données (ou ``contraintes'') qui sont 
des données servant à produire le programme.
Cette conception de la programmation basée sur l'optimisation
d'un tableau de paramètres via un ensemble de contraintes
remonte historiquement au principes de la programmation dynamique \shortcite{Bellman1956}.

Le but d'un algorithme d'apprentissage est de définir 
le jeu de paramètres $T$ de manière à produire une sortie 
conforme au calcul souhaité. 
Le jeu de paramètres est assimilables à un \textit{programme} et
le choix de ces paramètres correspond à la programmation du réseau. 

L'apprentissage automatique 
s'exprime donc principalement
sous la forme d'un problème d'optimisation dans des espaces vectoriels de grande dimension.  Il s'agit  d'optimiser une fonction de réponse $f_T$, définie selon un tableau de paramètres $T$, selon un critère d'optimalité défini par une fonction de coût $\mathcal{H}$ basée sur un jeu de contraintes $\mathbf{D}$~:
\begin{equation}
\min_T \mathcal{H}(T,\mathbf{D})
\end{equation}
où $T$ est le ``programme'' et $\mathbf{D}$ les ``méta-données''.
 
Dans le cadre de l'apprentissage automatique, les contraintes $\mathbf{D}$ sont les données d'apprentissage, et prennent la forme d'une base d'exemples  organisée sous forme de couples $\{(x_1,y_1),...,(x_n,y_n)\}$.  L'optimisation consiste à trouver une fonction de l'espace des entrées $\mathcal{X}$ vers l'espace des sorties $\mathcal{Y}$ qui minimise la distance aux données. 


L'optimisation par énumération est généralement impossible pour les problèmes de grande dimension. On a alors recours à des algorithmes dits d'''optimisation stochastique'', utilisant des hypothèses simplificatrices (comme la séparabilité linéaire des données ou encore le caractère Gaussien de distributions génératrices).  
 Le but de l'algorithme d'apprentissage est alors d'extraire des données fournies des régularités non apparentes, des facteurs explicatifs,
	permettant de mieux prédire et classifier les données nouvelles.

Exemples:
\begin{itemize}
	\item Les approches discriminatives \shortcite{Vap95}
	extraient des représentants caractéristiques des données (vecteurs supports). {\color{Cyan} Vecteurs les plus discriminants.}
	\item Les approches prédictives \shortcite{Bishop2006}
	identifient la source cachée des observations (les facteurs causaux) selon des modèles probabilistes pré-définis. {\color{Cyan} Vecteurs caractéristiques d'une classe.}
	\item Les réseaux de neurones (voir section suivante), 
	dont les caractéristiques propres (nombre de neurones, nombre de couches cachées, architecture
	et patron de connexions), déterminent une
	méthode d'optimisation spécifique.
\end{itemize}


\subsubsection{Lien entre le computationnel et le biologique}

De nombreux points communs peuvent être établis entre l'apprentissage computationnel et  l'apprentissage biologique. L'apprentissage automatique identifie en effet des \textit{communautés de problèmes}:
\begin{itemize}
	\item définissant des contraintes sur le type de solutions attendues
	\item nécessitant souvent des méthodes de résolution spécifiques.
\end{itemize} 
%Les problèmes identifiés dans le cadre computationnel sont souvent transposables au cas biologique. Dans ce cadre, 
%\begin{itemize} 
%	\item les contraintes imposées au 
%	système biologique sont du même ordre
%	\item les mécanismes de résolution peuvent présenter des points communs avec ceux qui sont mis en œuvre dans un cadre artificiel (\textit{communautés de mise en œuvre}).
%\end{itemize}

{\bf Remarque :} la communauté de problèmes n'implique pas nécessairement la \textit{ communauté de mise en œuvre}. Comme nous avons vu dans l'introduction~:
\begin{itemize}
	\item identifier correctement les problèmes et les contraintes des systèmes biologiques sont des enjeux à part entière
	\item il existe des contraintes spécifiques aux systèmes biologiques, telles que la contrainte de localité (plasticité de Hebb)  
	\item Rien n'indique à l'heure actuelle que les méthodes mise en œuvre soient les mêmes. Les méthodes d'optimisation développées en apprentissage automatique peuvent servir de guide à l'interprétation des données biologiques, mais sont souvent fort éloignées de la réalité biologique.  
\end{itemize}


\paragraph{Reconnaissance de formes et apprentissage}

Ce qui différencie principalement le calcul traditionnel du calcul distribué est la taille et la dimension des opérandes. Le calcul distribué s'adapte bien à la réalisation d'opérateurs (a priori non linéaires) dans les espaces de grande dimension. 

Le mécanisme fondamental sur lequel s'appuient la plupart des algorithmes d'apprentissage automatique est celui le l'\textit{appariement} (``\textit{pattern matching}'') entre un \textit{dictionnaires de formes} et un vecteur (ou un ensemble de vecteurs) d'entrée. 
Cet appariement se réalise:
\begin{itemize}
	\item à l'aide d'un produit scalaire
	\item par un calcul de distance
	\item par accumulation d'évidence (identification de composants caractéristiques)
	\item par une dynamique de relaxation (réseau de Hopfield)
	%\item par prédiction et interpolation
	\item etc.
\end{itemize}

La plupart des algorithmes d'apprentissage consistent à construire des dictionnaires de formes qui servent de clé d'interprétation des données d'entrée. 
Le choix de ces formes caractéristiques est bien sûr dépendant des capacités d'expression de la fonction de réponse~:
\begin{itemize}
	\item nombre de formes (taille du dictionnaire)
	\item types de dépendances pouvant être capturées (dépendances entre classes, dépendances spatiales, hiérarchies, dépendances temporelles, etc.)
\end{itemize}

{\bf Remarque~:} contrairement aux approches de traitement du signal classique (décomposition du signal sur des bases propices à l'analyse), les dictionnaires de formes sont ici entièrement \textit{construits} à partir des contraintes, c'est à dire principalement le jeu de données fourni.



{\color{Gray}Reconnaissance de formes. Feedback positif / feedback négatif. Mémoire associative. interpolation. 

Résonance.}



\paragraph{Types de problèmes}

La fonction de réponse $f_T$ est en général une projection sur un espace $\mathcal{Y}$ de dimension plus petite que l'espace d'entrée $\mathcal{X}$.
La fonction opère donc une \textit{réduction} de son espace d'entrée, c'est à dire qu'elle réduit les données d'entrée à un petit nombre de caractéristiques. C'est donc un transformation non-inversible du signal d'entrée avec perte d'information.

La nature de cet espace de sortie (continu ou discret) définit deux grands types de problèmes d'apprentissage. 

\begin{itemize}
	\item Lorsque l'espace de sortie est discret, la réponse est une \textit{catégorie}. On parle de catégorisation ou de classification des données d'entrée. 
	Dans ce cas, la fonction de réponse repose sur un mécanisme d'\textit{appariement exclusif}, c'est à dire qu'en fin de compte, une forme unique est identifiée comme correspondant aux données d'entrée. Ce mécanisme est mieux connu sous le nom de ``winner takes all'' (le gagnant prend tout), implémenté par la fonction \textit{argmax}.
	\item Lorsque l'espace de sortie est continu, on parle de problème de régression (ou de prédiction). La réponse est en général le résultat d'une \textit{décomposition} des données d'entrée sur une base de plus petite dimension. On parle dans ce cas d'\textit{appariement non exclusif} puisque les données d'entrée s'apparient à des degrés divers aux différentes formes de la base pour produire la réponse. 
\end{itemize}

{\color{Cyan} Remarque~: projection dans un espace de pls grande dimension \--- espace de redescription.}

\paragraph{Types de données}

La construction du dictionnaire de formes s'appuie sur des contraintes exprimées dans les données d'apprentissage. On distingue en général l'apprentissage guidé par l'erreur (supervisé ou semi-supervisé) et l'apprentissage guidé par le modèle.
\begin{itemize}
	\item L'apprentissage est guidé par l'erreur lorsqu'il existe une fonction d'erreur (ou fontion de ``\textit{feedback}'') $l$  qui ``note'' la réponse produite par la fonction de réponse. Exemples~:
	\begin{itemize}
		\item  fonction différence~: $l(y,f_T(x)) = y - f_T(x)$
		\item fonction distance~:
		$l(y,f_T(x)) = ||y - f_T(x)||$
		\item fonction de perte (ou  ``\textit{loss}'')
		$l(y,f_T(x)) = \mathbf{1}_{y \neq f_T(x)}$
		\item etc...
	\end{itemize}
	Le guidage est plus ou moins contraignant selon le type de fonction d'erreur considéré. 
	\begin{itemize}
		\item On parle d'apprentissage \textit{supervisé} (ou à information totale) lorsque la réponse désirée est connue,
		\item et d'apprentissage \textit{par renforcement} (ou à information partielle) lorsque la fonction d'erreur ne fournit qu'une ``indication'' (de type ``bien/mal) sur la justesse de la réponse.  
	\end{itemize} 
	\item L'apprentissage est guidé par le modèle lorsque les capacités d'expression de la fonction de réponse (principalement le nombre et le type de formes \---~ou classes~\--- identifiables) contraignent l'espace des réponses possibles. On parle d'apprentissage \textit{non supervisé}.
\end{itemize}
Le principe d'un guidage plus ou moins contraignant se retrouve également dans la littérature de psychologie expérimentale classique, dans laquelle on parle~:
\begin{itemize}
	\item de conditionnement pavlovien (lorsque le comportement est guidé par une réponse réflexe à un stimulus) {\color{Cyan}[PAVLOV]}
	\item et conditionnement opérant (lorsque le comportement est guidé par une récompense) {\color{Cyan}[SKINNER]}. 
\end{itemize} 

\paragraph{Apprendre et oublier}

Une dernière distinction concerne le caractère stationnaire ou non-stationnaire des données d'apprentissage lorsque le jeu de données est indexé sur l'axe temporel. 

On parle d'apprentissage ``en ligne'' lorsque les données sont présentées séquentiellement.
L'apprentissage consiste à appliquer de petits ajustements à chaque nouvelle donnée dans un 
sens qui augmente localement la justesse de la fonction de réponse, en espérant que chacun de ces ajustements locaux 
augmente la justesse globale,
selon le principe de la descente de gradient ``stochastique''.

Ce mécanisme d'apprentissage en ligne a l'avantage de rester valide lorsque les données sont non-stationnaires Ainsi les régularités observées durant les premières phases de l’apprentissage ne sont plus nécessairement présentes
à une étape ultérieure de l'apprentissage.
%~: certaines trajectoires sont ``abandonnées''. 
Dans une perspective de parcimonie 
(ou de ressource limitée), il est avantageux d'oublier certains faits passés pour mieux appréhender les faits nouveaux, autrement
dit oublier pour mieux apprendre.
Cette prise en compte du ``vieillissement'' de l'environnement conduit à accorder plus de crédit à des observations
récentes qu'aux observations anciennes \shortcite{Kivinen2004}, et donc \textit{oublier pour mieux apprendre}. 


%{\color{Cyan}
%\subsubsection{Perspective psychologique}
%Le conditionnement Pavlovien.

%L'approche constructiviste (Piaget).
%}


\subsection{Réseaux de neurones et apprentissage}

Il est facile d'établir une correspondance formelle entre l'apprentissage (au sens informatique) et la plasticité, en identifiant~:
\begin{itemize}
	\item les paramètres de la fonction de réponse aux poids synaptiques,
	\item et la plasticité à un mécanisme de mise à jour des paramètres guidé par les données.
\end{itemize} 
Ce cadre général est celui des réseaux de neurones, vus dans le chapitre précédent: la fonction de réponse est implémentée sous la forme d'un graphe : les noeuds sont les ``neurones'' et les arêtes sont les ``synapses''. Les réseaux de neurones et leurs méthodes de programmation sont un des piliers de la recherche 
en apprentissage automatique.    

Dans ce cadre, 
les réseaux de neurones sont principalement utilisés pour
des tâches de catégorisation de leurs données d'entrée.
Les méta-données sont alors constituées 
d'un jeu de données d'entrée ainsi que des sorties attendues,
appelé \textit{base d'apprentissage}.
La plupart des algorithmes d'apprentissages s'affranchissent de la contrainte de localité définie par Hebb et implémentent des mécanismes de mise à jour basés sur la propagation d'erreur (et non sur la conjonction d'activité).  

% programme = paramètres
% algorithme = meta programme



\subsubsection{Modèles à couches}

Les modèles de réseaux de neurones, % (encore dite approche ``distribuée''), 
tels qu'il se sont développés depuis le perceptron \shortcite{Rosen58}, reposent majoritairement sur
un principe de traitement séquentiel (``\textit{feed forward}'')
[REFERENCES : PMC, Vapnik, LeCun, ].
%une notion implicite de ressource limitée.

{\color{Cyan} Dans le calcul de la fonction de réponse $f_T$, les couches interédiaires implémentent les résultats de calcul intermédiaires.}


\begin{itemize}
	\item Le modèle de perceptron le plus simple, possédant une cellule de sortie unique, sépare son espace d'entrée en deux régions.
	Les données provenant de la première région produisent une activité de sortie ``haute'', qui s'apparente
	au {\color{Orange} mode ``actif''}. Les données provenant 
	de la seconde région produisent une une activité de sortie ``basse'', qui s'apparente au {\color{Orange} mode ``inactif''}.
	\item Dans le cas du perceptron multi-couches [CITATION],
	chaque neurone de la couche cachée est un séparateur permettant de distinguer deux régions de son espace d'entrée. 
	{\color{Orange} Le nombre de régions séparables (le nombre maximal de modes de sortie) augmente exponentiellement avec le nombre de neurones dans la couche  cachée.}
	Le concepteur fixe à l'avance le nombre de neurones à mettre dans la couche cachée,
	et donc la complexité des opérations de ségrégation produites par le réseau.
	\item Enfin, le modèle des séparateurs à vaste marge (SVM) \shortcite{Vap95}
	propose une définition quantitative de la complexité d'un classifieur~: la
	dimension de Vapnik-Chervonenkis, qui correspond au nombre de pivots utilisé pour
	séparer les données d'entrée. 
\end{itemize}

L'organisation d'un réseau à couches correspond à la transformation d'un signal d'entrée par étapes successives pour produire un signal de sortie. Les couches dites ``cachées'' correspondent à des étapes de calcul intermédiaires. 
La programmation des modèles à couches repose sur un principe de \textit{propagation du signal d'erreur} dans le sens inverse de la propagation des données, de la sortie vers l'entrée [REFERENCE BACKPROP].  

\subsubsection{Modèles récurrents}
{\color{Cyan}
Cas des RN récurrents. Règle de Hebb et mémoires associatives.

Assemblées neuronales et règle de Hebb. Intégration

Architectures récurrentes et séquences (patrons spatio-temporels)


Modèles récurrents et mémoire. En particulier mémoire associative. Les ``faits''. 
}


\subsection{Perspective computationnelle de la plasticité de Hebb}

La plasticité de Hebb est, comme nous l'avons vu, une règle de modification locale du graphe fondée sur les conjonctions d'activité pré- et post-synaptiques. 
La règle de Hebb est une règle essentiellement additive et s'interprète comme la règle de ``\textit{plus du même}''. 
En pratique, le mécanisme facilitateur de Hebb est couplé avec un mécanisme stabilisateur (soustractif) qui évite la divergence du processus (via un principe de sélection compétitive) \cite{abbott00}.  



Les implémentations les plus connues du principe de Hebb sont~:
\begin{itemize}
	\item mémoire associative de Hopfield
	\item codes auto-correcteurs
	\item réseau ART
	\item cartes de Kohonen
\end{itemize}


{\color{Gray}Hebb a été influence par la psychologie expérimentale de son époque et transpose les principes du conditionnement opérant au cadre microscopique.   }

%Dans une perspective computationnelle, si la fonction d'activation $f$ définit le programme, la fonction de mise à jour des poids $F$ agit sur le programme implémenté par $f$. La fonction $F$ peut donc être vue comme un mécanisme de programmation du calculateur $f$. 

%Approche empirique: reproduire l'apprentissage via la plasticité sur un support artificiel.

%en modifiant son activité, sa réactivité à certaines entrées, ou le type de
%réponse qu'il produit. 


{\color{Orange} Le but est d'implémenter via la plasticité une mémoire ``persistante'', qui provoque un changement durable de la fonction de réponse.
	On souhaite inscrire sur
	le graphe de connexions la trace de faits passés en vue d'une utilisation future.}

%{\bf !!Plaçons-nous par exemple dans une perspective computationnelle autonome!!}, 
%dans laquelle un programme doit pouvoir se passer d'utilisateur. 
%La consigne doit provenir d'un programme interne (un ``méta programme'') qui 
%agit sur le programme courant pour l'``améliorer'' en fonction des événements se présentent.
%{\bf !!Même si la notion de méta-programme ne pose pas de difficulté conceptuelle, elle est
%	en pratique difficile à mettre en oeuvre dans les architectures informatiques traditionnelles. 
%	La difficulté réside dans 
%	le choix des événements à considérer parmi l'ensemble des événements qui se présentent, et surtout 
%	dans la manière dont les événements pris en compte influencent le comportement logiciel futur.}

%Il existe deux grandes approches de la plasticité~:
%\begin{itemize}
%	\item Apprentissage sans consigne.
%	\item Apprentissage avec consigne
%\end{itemize}


\section{Notions spécifiques et propositions}

{\color{Cyan}
	Idée de base~: étudier le rôle de l'activité récurrente (persistante) dans le cadre de l'apprentissage. En particulier: reconnaissance de patrons spatio-temporels. 
	
	Mécanisme d'appariement distribué: l'opérateur est l'opérande et subit les transformations.   
	
	Le modèle qui est proposé, c'est principalement une alternance entre (i) absence de réponse et (ii) mise en œuvre de programmes moteurs spécifiques.
	
	Ce fonctionnement est lié au principe d'assemblée neuronale et à la plasticité de Hebb, à travers la notion  de sélection (au sein d'un réservoir).
	
	On peut étendre à l'idée de vocabulaire comportemental/appariement sélectif/''winner takes all''/argmax.
	
	Modèle de mémoire basé sur l'interprétation suivante ~: les formes mémorisées sont des ``faits''. }

\subsection{Assemblée neuronale}
Le principe de plasticité par conjonction d'activité est articulé, dans le livre de Hebb, avec le concept d'\textit{assemblée neuronale} \cite{Heb49}. 
Une assemblée neuronale est un sous-ensemble de nœuds du réseau s'activant de manière coordonnée, et de façon réitérée au cours du temps.
La notion d'assemblée est donc liée à la notion d'activité~: une assemblée est un patron d'activité qui réapparaît fréquemment (l'activité d'un nœud étant identifiée à un état ``haut'').

{\color{Gray} La notion d'assemblée se distingue de la notion de population~: 
	\begin{itemize}
		\item les populations de neurones sont définies par des caractères architecturaux
		\item les assemblées de neurones sont définies par la dynamique.
	\end{itemize} 
	
	On identifie fréquemment l'activation d'une assemblée à la réalisation d'une tâche spécifique.}

Une assemblée est donc, selon l'idée originale de Hebb, un patron d'activité caractéristique, sélectionné (stabilisée) par la règle de plasticité. 
{\color{Gray} D'un point de vue computationnel, cette
assemblée de neurones actifs est une forme (un vecteur caractéristique)  ``stockée'' dans le réseau, qui a la possibilité de s'''activer'' en fonction du contexte.}
On parlera dans la suite, pour désigner cette activité coordonnée, d'une activité \textit{macroscopique} (par opposition à l'activité microscopique du neurone seul). 

Une assemblée de neurones peut être vue comme une unité de description de l'activité du réseau.
A un instant donné, une assemblée se trouve soit dans l'état inactif, soit dans l'état ``activé''. L'état du réseau est alors décrit, de manière macroscopique, par la (ou les) assemblée(s) particulière(s) qui est (sont) active(s) à cet instant.

\subsubsection{Activité récurente}


	Le principe d'assemblée neuronale s'interprète principalement dans le cadre des réseaux de neurones récurrents. 
	Nous avons vu que les réseaux de neurones récurrents sont capables de développer une activité propre (activité ``endogène''), c'est à dire que les neurones peuvent demeurer dans un état ``actif'' en absence de stimulation extérieure. Cette activité interne repose sur la transmission ``en boucle'' de l'activité des neurones sur le graphe. 
	
	Dans ce cadre, une assemblée peut être vue comme une séquence d'activation interne stable au cours du temps. On parle parfois de ``chaîne d'activation'' \--- \textit{synfire chain} \shortcite{Abe93}. Un certain nombre de modèles artificiels implémentent ce principe, comme le modèle des  ``groupes polychrones'' 
	postulé par \shortcite{Izh06}.
	
{\color{Gray}	
	pouvant présenter des régularités liée aux régularités du graphe ou liée à une influence extérieure (activité induite). 
	%La règle de Hebb s'interprète dans ce cadre comme le fait de capturer dans la structure du réseau les régularités statistiques (covariances) présentes dans l'activité spontanée. 
	Une assemblées apparaissant fréquemment tend à s'inscrire dans le graphe de connexions, ce qui tend à favoriser sa réapparition dans le futur.
}

Le modèle mathématique qui implémente le principe d'assemblée de la manière la plus simple est le modèle de Hopfield \shortcite{Hop82}. Ce modèle est un réseau de neurones récurrent à unités binaires développant une dynamique de point fixe multistable. Le réseau de Hopfield implémente un principe de \textit {mémoire associative}. Il apprend ``par cœur'' un certain nombre de scènes (ou images). Il est ensuite capable, lorsque certains constituants d'une scène apprise sont présentés, de compléter la scène avec les éléments manquants, de rectifier les valeurs erronées etc. 
L'espace d'états est organisé en un nombre fini de \textit{bassins d'attraction} définissant des patrons d'activité spécifiques et mutuellement exclusifs. L'activité du réseau finit par converger vers un de ces patrons en fonction des conditions initiales. Chacun des attracteurs s'interprète ici comme une assemblée distincte, c'est à dire comme un circuit de renforcement mutuel aboutissant à la stabilisation d'un patron d'activité macroscopique. 


\subsection{Appariement distribué et intégration}

D'un point de vue computationnel, le modèle de Hopfield 
implémente un mécanisme d'appariement entre un signal d'entrée, 
fourni sous forme de condition initiale, et un des patrons stockés dans le graphe.
Contrairement aux modèles classiques dans lesquels l'appariement 
repose sur un produit scalaire, l’appariement s'effectue ici via  mécanisme de relaxation vers un attracteur. Le calcul de la réponse (le patron d'activité final) repose sur la stabilisation progressive d'un circuit d'activation particulier. On parle dans ce cas de processus  d'\textit{intégration} de la réponse (au sens de l'intégration progressive au cours du temps de caractéristiques d'activation conduisant au patron final).

Ce mécanisme d'intégration est spécifique aux réseaux de neurones récurrents, dans lequel des valeurs d'activation calculées aux temps précédents sont réutilisées pour aboutir au patron final.  
Il n'y a donc pas, dans ce cadre, de distinction entre opérateur et opérande. L'information ne circulant pas dans un sens défini (de l'entrée vers la sortie), l'activité d'un nœud est, selon l'avancement du processus d'intégration, un opérande (une entrée) ou le résultat d'une opération.

Le résultat macroscopique de ce processus est un mécanisme d'accroissement d'évidence, inscrit dans le temps, conduisant 
au ``choix'' d'une des formes préalablement stockées dans le réseau. 

Malgré des limitations bien connues en terme de capacité de stockage \shortcite{Amit1985}, le réseau de Hopfield présente l'avantage de fonctionner avec des principes compatibles avec les idées de Hebb (caractère local de la règle de plasticité, analogie possible entre attracteur et assemblée), et donc de constituer un modèle plus ``biologiquement plausible'' que les réseaux de neurones traditionnels.


\subsection{Réduction de la dynamique}

Comme nous l'avons vu, la plupart des algorithmes d'apprentissage fonctionnent selon un principe d'\textit{appariement} (``pattern matching''), c'est à dire sur la construction, à partir des données, d'un ensemble de vecteurs ``caractéristiques''. Ces vecteurs caractéristiques, vus comme un dictionnaire de formes, constituent une base de description synthétique des données d'entrée. La projection des données d'entrée sur cet espace de description facilite ensuite leur traitement et leur interprétation.  L’apprentissage revient donc à la constitution d'un vocabulaire de base qui sert de clé d'interprétation des données d'entrée. 

La transformation des données d'entrée sous la forme d'un (ou plusieurs) vecteur(s) caractéristique(s) correspond à une   \textit{simplification} de ces données.
Cette simplification se traduit par exemple, dans le cas des réseaux de neurones à couches, par une dimension de la couche de sortie plus faible que celle de la couche d'entrée.
Dans le cas des réseaux récurrents, cette simplification se traduit par une \textit{réduction du nombre de degrés de liberté} sur lesquels évolue la dynamique des neurones. L'activité du réseau accepte alors une description sur un espace de plus petite dimension.
    
Cette idée de réduction de la dynamique a été conceptualisée par les travaux de Haken XXX, qui reprend la notion de variable d'ordre issue de la thermodynamique dans le cadre de la description des systèmes biologiques. Selon l'interpétation de Haken, 


Complexité, chaos et réduction de la complexité.

Principe de sélection. La réponse ``je ne sais pas''.


\subsection{La réponse ``je ne sais pas''}

Freeman

\subsection{Le principe du réservoir}



\subsection{Principe d'appariement sélectif}



{\color{Cyan} 
	Exemples :
	\begin{itemize}
		\item Appariement sélectif~: les barycentres des partitions génératrices permettant d'interpréter une donnée d'entrée comme membre d'une classe donnée. Principe de l'argmax ou du ``winner takes all''. Non-linéaire. \textit{Enumeration.}
		\item Appariement non sélectif : les données sont interprétées comme une combinaison de facteurs.  projection des données sur une base de plus petite dimension. Matching pursuit. Dictionnaire. Combinaison linéaire. ACP, SVD, deep learning, ... \textit {Décomposition}.
	\end{itemize}
}

%finissent la manière dont les données vont être traitées et interprétées.


\subsection{Sélection/intégration}

Mécanismes de choix qui sélectionne les signaux et les événements plus significatifs;

Passage de seuil. ``Spike'' de population. De la cellule à la population. 
Réponse de population induite.

(Freeman) et  sélection/intégration  par réduction

Calcul collectif - intégration - assemblée

Mecanisme de choix. Lien entre choix et plasticité. On reconnait ce que l'on connait (Reconnaisance et connaissance). analogie avec spike de population (spike = significatif) 



sélection = Analogie accumulation d'évidence?

\subsection{Construire les modes de la fonction de réponse}

Une des thèses du mémoire : feed+/feed-.

\section{Contributions personnelles}

\chapter{Architectures de contrôle}

\section{Notions générales}

Langage de l'action. 

Espace de la tâche. Passage d'échelle et variable d'ordre.

Scène sensorielle et scène motrice

\section{Notions spécifiques et propositions}

\section{Contributions personnelles}

\chapter{Projet}

{\color{Cyan} Le cerveau n'est pas une machine de traitement de l'information mais une machine de production de sens.}


%@article{held1974development,
%	title={Development of sensorially-guided reaching in infant monkeys},
%	author={Held, Richard and Bauer Jr, Joseph A},
%	journal={Brain research},
%	volume={71},
%	number={2},
%	pages={265--271},
%	year={1974},
%	publisher={Elsevier}
%}
%
%
%
%@article{jeannerod1995grasping,
%	title={Grasping objects: the cortical mechanisms of visuomotor
%		transformation},
%	author={Jeannerod, Marc and Arbib, Michael A and Rizzolatti, Giacomo
%		and Sakata, H},
%	journal={Trends in neurosciences},
%	volume={18},
%	number={7},
%	pages={314--320},
%	year={1995},
%	publisher={Elsevier}
%}
%
%
%
%@article{kerzel2003neuronal,
%	title={Neuronal processing delays are compensated in the
%		sensorimotor branch of the visual system},
%	author={Kerzel, Dirk and Gegenfurtner, Karl R},
%	journal={Current Biology},
%	volume={13},
%	number={22},
%	pages={1975--1978},
%	year={2003},
%	publisher={Elsevier}
%}
%
%@article{glover2004separate,
%	title={Separate visual representations in the planning and control of action},
%	author={Glover, Scott},
%	journal={Behavioral and Brain Sciences},
%	volume={27},
%	number={01},
%	pages={3--24},
%	year={2004},
%	publisher={Cambridge Univ Press}
%}
%
%
%
%@article{nijhawan2004compensation,
%	title={Compensation of neural delays in visual-motor behaviour: No
%		evidence for shorter afferent delays for visual motion},
%	author={Nijhawan, Romi and Watanabe, Katsumi and Khurana, Beena and
%		Shimojo, Shinsuke},
%	journal={Visual Cognition},
%	volume={11},
%	number={2-3},
%	pages={275--298},
%	year={2004},
%	publisher={Taylor \& Francis}
%}

\bibliographystyle{mslapa}%{apalike}%{apacite}

\bibliography{biblio}
\end{document}

